\section{Conclusions}
In this laboratory assignment I remembered the two main approaches when parallelizing a program using \omp. The worksharing constructs model and the task based model. Which there is not much difference when parallelizing loops but what happens begin (\omp runtime) is different.

\justify
Also I learned that when improving the performance of a program, is very important to improve the performance of the base code, and then parallelize it in order to get better speedups.

\justify
Another thing I noticed is that the task based model has a big impact on work distributing overheads, as it has been noticed in the big difference of performance with the two approaches in the first versions while it's not noticeable in the improved version where we just need to parallelize a loop and actually the task based model performs better.

