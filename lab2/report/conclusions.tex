\section{Conclusions}

In this laboratory assignment I learned how the OpenMP internals work. I got a high level point of how from a pragma we actually run parallel code.

\par ~\\
In particular, I got to understand how the tasking model works in OpenMP, from where the overheads come, and what the trade offs are when using tasks.

\par ~\\
Another important part of this work, was writing parallel code using pthread standard library, which gave me a more clear understanding of producer consumer scheme, parallel programming thinking, specifically, race conditions.

\par ~\\
During the development of this laboratory assignment I also improved various skills like bash scripting, gnuplot, performance evaluation, writing of scientific documents, skills that I have not much experience and I mostly learned by myself.

\par ~\\
Finally I will explain the things I am proud about and the things can be improved.

\par ~\\
About the things can be improved, I think the implementation of the miniomp has to be reviewed, in part the complete architecture, and another part review the synchronisation on the circular buffer, because it may contain synchronisation not needed. The other main thing it has to be improved is this document, which have been written with lack of time and have to be reviewed. The last thing have to be improved is the performance evaluation which lacks of evaluation on different programs.
\par ~\\
I'm proud that my implementation performs similar to OpenMP implementation although it's restricted. I'm also proud of my script for scalability analysis which is expandable, capable of analysis of multiple versions. I'm very happy of the final result despite the lack of time.

